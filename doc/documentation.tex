\documentclass{article}
% some package imported, idk if I'll use all of it, otherwiser I will remove them
\usepackage{amsmath} % import of math elements
\usepackage{mathtools} %import of other math elements
\usepackage{tikz}
\usetikzlibrary{shapes,positioning,calc} 

%-------------------------------------------------------
% Document information
%-------------------------------------------------------

\author{
  Roberto Antoniello \ \ \ \ \ \&
  \and
  Edoardo Ferrari} %author name
  
\title{eMerger Documentation} %Title 

\begin{document}
\maketitle % show the title and author and date
%-------------------------------------------------------
%Introduction
%-------------------------------------------------------

\begin{center}In this file we will put a complete documentation of eMerger. So you can read simply how it really works without waste too much time reading only the code.
\end{center}

\section{Introduction}
We start this project for fun without knowing a bit of Bash, but then we have continued to release new features and improve the main script, also learning new stuffs during the development.\\
The project started with the name of \textit{Updater}, then we changed to \textit{eMerger}(*blink Gentoo dudes).\\
Probably you already know if you're reading this but let's repeat what is eMerger.\\
eMerger is a script that allows users to do a clean update for Linux based system with a single command. 

\section{Installation}
The only thing you need to do to install eMerger is to launch \textit{setup.sh}.\\
When you launch it for the first time, it just put an alias on your ~/.bashrc named \textit{up} pointed on the source file \textit{emerger.sh}. \\
After the alias is added correctly  or the alias already exists, it will launch an integrity test by calling the source file \textit{integrity-check.sh}.\\
During the installation a little cache is also created so eMerger will always remember which system you're using without fetching this data every time.

\subsection{Integrity test}
In this script there's a check of the existence and stability of the content of source files needed for eMerger correct execution. If it's all ok we continue, otherwise the current operation is aborted.

\subsection{Uninstall}
The script \textit{uninstall.sh} just removes the alias created during the installation and delete every file related to eMerger.
\section{Functioning}




\end{document}